\usepackage{fontspec}
\usepackage[babelshorthands=true]{polyglossia}

\setdefaultlanguage{russian}
\setotherlanguage{english}


\defaultfontfeatures{Ligatures=TeX}

\setmainfont{Times New Roman}
\setsansfont{Arial}
\setmonofont[Contextuals=Alternate, Scale=MatchLowercase]{Fira Code}

\newfontfamily\cyrillicfont{Times New Roman}
\newfontfamily\cyrillicfontsf{Arial}
\newfontfamily\cyrillicfonttt[Contextuals=Alternate, Scale=MatchLowercase]{Fira Code}

\newfontfamily\englishfont{Times New Roman}
\newfontfamily\englishfontsf{Arial}
\newfontfamily\englishfonttt[Contextuals=Alternate, Scale=MatchLowercase]{Fira Code}

\usepackage{microtype} % nicer layouts with microtypography tricks
\usepackage{fnpct} % footnotes after punctuation look nicer

\usepackage[
    inner = 1.5cm,
    textwidth=345pt,
    bottom = 1.5cm
]{geometry}

\usepackage{enumitem} % russian style lists
\setlist{nosep}
\setlist[enumerate]{label={\arabic*)}}
\setlist[itemize]{label={---}}

\usepackage[outputdir=build]{minted}
\usemintedstyle{vs}

\newcommand{\inlcpp}[1]{\mintinline[breaklines, breakanywhere]{c++}{#1}}
\newcommand{\mintlbl}[1]{\phantomsection\label{#1}}

\usepackage{amsfonts}

\usepackage[dvipsnames]{xcolor}
\usepackage{xurl}
\urlstyle{same}
\usepackage[hidelinks,colorlinks=true]{hyperref}
\hypersetup{
  linkcolor=BrickRed,
  citecolor=Green,
  filecolor=Mulberry,
  urlcolor=NavyBlue,
  menucolor=BrickRed,
  runcolor=Mulberry,
  linkbordercolor=BrickRed,
  citebordercolor=Green,
  filebordercolor=Mulberry,
  urlbordercolor=NavyBlue,
  menubordercolor=BrickRed,
  runbordercolor=Mulberry
}

\usepackage[backend=biber,
  bibencoding=utf8,
  sorting=none,
  style=gost-numeric,
  language=autobib,
  autolang=other,
  clearlang=true,
  defernumbers=true,
  sortcites=true,
]{biblatex}

\usepackage{easy-todo}

\overfullrule=10pt
